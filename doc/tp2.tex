
%	Documentação do Trabalho Prático 1 de AEDSIII
%	@Sandro Miccoli
%
%	* Você pode identificar erros de grafia através do seguinte comando linux:
%		aspell --encoding="utf-8" -c -t=tex --lang="pt_BR" tp1.tex
%

\documentclass[12pt]{article}
\usepackage{sbc-template}
\usepackage{graphicx}
\usepackage{latexsym}
\usepackage{subfigure}
\usepackage{times,amsmath,epsfig}
\usepackage{graphicx,url}
 \makeatletter
 \newif\if@restonecol
 \makeatother
 \let\algorithm\relax
 \let\endalgorithm\relax
\graphicspath{{./data/}}
\usepackage[lined,algonl,ruled]{algorithm2e}
\usepackage{multirow}
\usepackage[brazil]{babel}
\usepackage[utf8]{inputenc}
\usepackage{listings}

\usepackage{alltt}
\renewcommand{\ttdefault}{txtt}

\sloppy

\title{TRABALHO PRÁTICO 2: \\ Paradigma Guloso e Programação Dinâmica}

\author{Sandro Miccoli - 2009052409 - smiccoli@dcc.ufmg.br}

\address{Departamento de Ciência da Computação -- Universidade Federal de Minas Gerais (UFMG)\\
\\
\today}


\begin{document}

\maketitle

\begin{resumo}
Este relatório descreve como foi solucionado o problema de encontrar o menor palíndromo possível dada uma palavra. Será descrito como foi modelado o problema, pois duas soluções foram propostas, uma abordagem de paragima guloso e outra de programação dinâmica. Finalmente será detalhado a análise de complexidade dos algoritmos e uma análise comparativa entre os dois paradigmas e uma breve conclusão do trabalho implementado.
\end{resumo}

\section{INTRODUÇÃO}

    Um palíndromo é uma palavra, frase ou qualquer outra sequência de unidades que tenha a propriedade de poder ser lida tanto da direita para a esquerda como da esquerda para a direita \cite{wikipalin}.

    O problema deste trabalho é encontrar qual o menor palíndromo possível de ser gerado inserindo caracteres na palavra original. No caso da palavra CASA, por exemplo, a solução seria CASAC.

    Para solucionar essa questão foi criada duas abordagens, uma de solução ótima, utilizando programação dinâmica (PD) e outra de solução não-ótima, utilizando o paradigma guloso (PG).

	O restante deste relatório é organizado da seguinte forma. A Seção~\ref{modelagem} descreve como foi feita a modelagem do problema e manipulação das palavras. A Seção \ref{solucao_proposta} descreve as propriedades de sub-estrutura ótima e sobreposição dos problemas do paradigma guloso, e também discute a otimalidade da solução gulosa. A Seção~\ref{implementacao} trata de detalhes específicos da implementação do trabalho: quais os arquivos utilizados; como é feita a compilação e execução; além de detalhar o formato dos arquivos de entrada e saída. A Seção~\ref{avaliacao_experimental} contém a avaliação experimental, que faz uma análise de quantos porcento a mais de caracteres que a solução gulosa utiliza em relação à solução ótima. A Seção~\ref{conclusao} conclui o trabalho.


\section{MODELAGEM}
\label{modelagem}

	Para solucionar o problema do palíndromo, o modelamos como o problema da \textbf{Maior Subsequência Comum}, também conhecido como \textit{Longest Common Subsequence} (LCS). Uma subsequência de uma sequência de símbolos $X$ é uma sequência $Y$ com zero com maios símbolos de $X$ removidos. Exemplo: $Z = \{B, C, D, B\}$ é uma subsequência de $X = \{A, B, C, B, D, A, B\}$.
	\\
	Definição mais formal:\\
	\\
	\textit{Longest Common Subsequence (LCS)}: Datas as sequências $X [1..m]$ e $Y [1..n]$ encontrar uma sequência $Z$ que seja subsequência de $X$ e $Y$ e tem comprimento (número de símbolos) máximo.\\
	$X = \{A, B, C, B, D, A, B\}$\\
	$Y = \{B, D, C, A, B, A\}$\\
	$LCS (X,Y) = \{B, C, B, A\}$\\

	Como o problema específico neste trabalho é encontrar palíndromos, então nossa sequência $Y$ é o inverso da nossa sequência $X$.

\section{SOLUÇÃO PROPOSTA}
\label{solucao_proposta}

\subsection{Abordagem Dinâmica}
\label{dinamica}

    Uma primeira abordagem escolhida para resolver este problema é o de Programação Dinâmica. Programação dinâmica é um método para a construção de algoritmos para a resolução de problemas computacionais, em especial os de otimização combinatória. Ela é aplicável a problemas nos quais a solução ótima pode ser computada a partir da solução ótima previamente calculada e memorizada - de forma a evitar recálculo - de outros subproblemas que, sobrepostos, compõem o problema original \cite{szwarccfiter}.

    O que um problema de otimização deve ter para que a programação dinâmica seja aplicável são duas principais características: subestrutura ótima e superposição de subproblemas. Um problema apresenta uma subestrutura ótima quando uma solução ótima para o problema contém em seu interior soluções ótimas para subproblemas. A superposição de subproblemas acontece quando um algoritmo recursivo reexamina o mesmo problema muitas vezes \cite{szwarccfiter}.

    O \textit{LCS} tem subestrutura ótima, pois pode ser quebrado em problemas menores, mais simples, os quais também podem ser divididos em subproblemas menores ainda, e assim por diante, até, finalmente, a solução for trivial. O \textit{LCS} também tem sobreposição de problemas: a solução de um subproblema maior depende da solução de vários subproblemas menores \cite{wikilcs}.

    Para calcular o \textit{LCS} utilizamos uma tabela para armazenar os valores calculados a cada passo da execução. As implementações mais encontradas na internet utilizavam uma tabela $(n+1, m+1)$, sendo $n$ e $m$ o tamanho de cada uma das strings. A abordagem dinâmica escolhida aqui foi utilizar uma tabela de tamanho $(n, 2)$, sendo $n$ o tamanho da \textit{string} que encontraremos o menor palíndromo, pois como o trabalho pede apenas o tamanho do \textit{LCS}, então só preciamos do estado atual e anterior de cada coluna da matriz, por isso preciamos de apenas duas linhas para realizar os cálculos \cite{hirsch}.

    A seguir um pseudocódigo do algoritmo implementado:

    \newpage

    \begin{lstlisting}[language=c]
    int LCS(X)
        M = array(n, 2) // Matriz de estados do LCS

        for i = n ate 0
            charRev = X[i] // Le palavra reversamente

            for j = 0 ate n
                charCon = X[j] // Le palavra continuamente

                    if charCon = charRev
                        M[i,j+1] = M[0,j] + 1
                    else
                        M[i,j+1] = max(M[1,j], M[0,j+1])

            for k = 0 ate n
                M[0][k] = M[1][k] // Atualiza estado atual

    return n - M[1,n]
    \end{lstlisting}


\subsubsection{Algoritmo implementado}

\begin{itemize}
 \item \begin{large}\textit{int lcs(char * palavra, int len, Matriz * estados)}\end{large}\\
 \subitem \textbf{Descrição:} Calcula o \textit{LCS} da palavra e retorna seu valor.
 \subitem \textbf{Parâmetros:} Vetor da palavra, inteiro contendo o tamanho da palavra, e uma matriz de estados.
 \subitem \textbf{Complexidade:} $O(n^2)$, onde $n$ é o tamanho da palavra.
\end{itemize}

\subsection{Abordagem Gulosa}
\label{gulosa}


\subsubsection{Algoritmo implementado}

\begin{itemize}
 \item \begin{large}\textit{int lcs(char * palavra, int len, Matriz * estados)}\end{large}\\
 \subitem \textbf{Descrição:} Calcula o \textit{LCS} da palavra e retorna seu valor.
 \subitem \textbf{Parâmetros:} Vetor da palavra, inteiro contendo o tamanho da palavra, e uma matriz de estados.
 \subitem \textbf{Complexidade:} $O(n^2)$, onde $n$ é o tamanho da palavra.
\end{itemize}

\vspace{0.2 true cm}


\section{IMPLEMENTAÇÃO}
\label{implementacao}

\subsection{Código}

\subsubsection{Arquivos .c}

\begin{itemize}
\item \textbf{tp2pd.c} Arquivo principal do programa com a abordagem de programação dinâmica, chama a função LCS e insere cada resultado no arquivo de saída.
\item \textbf{tp2g.c} Arquivo principal do programa com a abordagem gulosa, chama a função ? e insere cada resultado no arquivo de saída.
\item \textbf{lcs.c} Contém a implementação da função LCS, utilizada na abordagem de programação dinâmica.
\item \textbf{matriz.c} Contém todas as funções de manipulação, leitura e escrita de matrizes.
\item \textbf{arquivos.c} Um tipo abstrado de dados de manipulação de arquivos, contendo funções de abertura, leitura, escrita e fechamento.
\end{itemize}

\subsubsection{Arquivos .h}

\begin{itemize}
\item \textbf{lcs.h} Contém a definição da função LCS, utilizada na abordagem de programação dinâmica.
\item \textbf{matriz.h} Além de definir a estrutura de matriz, contém o cabeçalho todas as funções de manipulação, leitura e escrita de matrizes.
\item \textbf{arquivos.h} Definição da das funções utilizadas para ler, escrever e fechar corretamente um arquivo.
\end{itemize}

\subsection{Compilação}

O programa deve ser compilado através do compilador GCC através dos seguintes comandos

Pkara programação dinâmica:
\begin{footnotesize}
\begin{verbatim}
gcc -Wall -Lsrc src/tp2pd.c src/arquivos.c src/matriz.c src/lcs.c -o tp2pd \end{verbatim}
\end{footnotesize}

E para o algortimo guloso:
\begin{footnotesize}
\begin{verbatim}
gcc -Wall -Lsrc src/tp2g.c src/arquivos.c src/matriz.c -o tp2g \end{verbatim}
\end{footnotesize}

Ou através do comando $make$.

\subsection{Execução}

A execução do programa tem como parâmetros:
\begin{itemize}
\item Um arquivo de entrada contendo várias instâncias de palavras.
\item Um arquivo de saída que irá receber o resultado dos cálculos do menor palíndromo possível a ser gerado.
\end{itemize}

O comando para a execução do programa é da forma:

Programação Dinâmica:
\begin{footnotesize}
\begin{verbatim} ./tp2pd <arquivo_de_entrada> <arquivo_de_saída>\end{verbatim}
\end{footnotesize}

Paradigma Guloso:
\begin{footnotesize}
\begin{verbatim} ./tp2g <arquivo_de_entrada> <arquivo_de_saída>\end{verbatim}
\end{footnotesize}

\subsubsection{Formato da entrada}

A primeira linha do arquivo de entrada contém o valor \textit{k} de instâncias que o arquivo contém. As próximas $k$ linhas contém as palavras.

A seguir um arquivo de entrada de exemplo:

\begin{verbatim}
5
BOLO
XBOX
BANANA
SUCESSO
WOW
\end{verbatim}

\subsubsection{Formato da saída}

O arquivo de saída consiste em k linhas com o número mínimo de caracteres necessários para gerar um palíndromo com a palavra dada de entrada na linha correspondente. Um exemplo é mostrado abaixo:

\begin{verbatim}
1
1
1
4
0
\end{verbatim}


\section{AVALIAÇÃO EXPERIMENTAL}
\label{avaliacao_experimental}

Para testar o algoritmo realizamos quatro testes, separados em dois grupos. O primeiro grupo de testes se refere à grafos esparsos e o segundo a grafos densos.

\subsection{Máquina utilizada}
\label{maquina}

Segue especificação da máquina utilizada para os testes:
\begin{verbatim}
model name:     Intel(R) Core(TM) i3 CPU       M 330  @ 2.13GHz
cpu MHz:        933.000
cache size:     3072 KB
MemTotal:       3980124 kB
\end{verbatim}


\subsection{Resultado}

	O resultado geral dos testes ocorreu como esperado.

\section{CONCLUSÃO}
\label{conclusao}

    Conclusão

\bibliographystyle{sbc}
\bibliography{tp2}

\end{document}
