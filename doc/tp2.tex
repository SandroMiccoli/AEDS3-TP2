
%	Documentação do Trabalho Prático 1 de AEDSIII
%	@Sandro Miccoli
%
%	* Você pode identificar erros de grafia através do seguinte comando linux:
%		aspell --encoding="utf-8" -c -t=tex --lang="pt_BR" tp2.tex
%

\documentclass[12pt]{article}
\usepackage{sbc-template}
\usepackage{graphicx}
\usepackage{latexsym}
\usepackage{subfigure}
\usepackage{times,amsmath,epsfig}
\usepackage{graphicx,url}
 \makeatletter
 \newif\if@restonecol
 \makeatother
 \let\algorithm\relax
 \let\endalgorithm\relax
\graphicspath{{./data/}}
\usepackage[lined,algonl,ruled]{algorithm2e}
\usepackage{multirow}
\usepackage[brazil]{babel}
\usepackage[utf8]{inputenc}
\usepackage{listings}

\usepackage{alltt}
\renewcommand{\ttdefault}{txtt}

\sloppy

\title{TRABALHO PRÁTICO 2: \\ Paradigma Guloso e Programação Dinâmica}

\author{Sandro Miccoli - 2009052409 - smiccoli@dcc.ufmg.br}

\address{Departamento de Ciência da Computação -- Universidade Federal de Minas Gerais (UFMG)\\
\\
\today}


\begin{document}

\maketitle

\begin{resumo}
Este relatório descreve como foi solucionado o problema de encontrar o menor palíndromo possível dada uma palavra. Será descrito como foi modelado o problema, pois duas soluções foram propostas, uma abordagem de paradigma guloso e outra de programação dinâmica. Finalmente será detalhado a análise de complexidade dos algoritmos e uma análise comparativa entre os dois paradigmas e uma breve conclusão do trabalho implementado.
\end{resumo}

\section{INTRODUÇÃO}

    Um palíndromo é uma palavra, frase ou qualquer outra sequência de unidades que tenha a propriedade de poder ser lida tanto da direita para a esquerda como da esquerda para a direita \cite{wikipalin}.

    O problema deste trabalho é encontrar qual o menor palíndromo possível de ser gerado inserindo caracteres na palavra original. No caso da palavra CASA, por exemplo, a solução seria CASAC.

    Para solucionar essa questão foi criada duas abordagens, uma de solução ótima, utilizando programação dinâmica (PD) e outra de solução não-ótima, utilizando o paradigma guloso (PG).

	O restante deste relatório é organizado da seguinte forma. A Seção~\ref{modelagem} descreve como foi feita a modelagem do problema e manipulação das palavras. A Seção \ref{solucao_proposta} descreve as propriedades de sub-estrutura ótima e sobreposição dos problemas do paradigma guloso, e também discute a otimalidade da solução gulosa. A Seção~\ref{implementacao} trata de detalhes específicos da implementação do trabalho: quais os arquivos utilizados; como é feita a compilação e execução; além de detalhar o formato dos arquivos de entrada e saída. A Seção~\ref{avaliacao_experimental} contém a avaliação experimental, que faz uma análise de quantos porcento a mais de caracteres que a solução gulosa utiliza em relação à solução ótima. A Seção~\ref{conclusao} conclui o trabalho.


\section{MODELAGEM}
\label{modelagem}

	Para solucionar o problema do palíndromo, o modelamos como o problema da \textbf{Maior Subsequência Comum}, também conhecido como \textit{Longest Common Subsequence} (LCS). Uma subsequência de uma sequência de símbolos $X$ é uma sequência $Y$ com zero com maios símbolos de $X$ removidos. Exemplo: $Z = \{B, C, D, B\}$ é uma subsequência de $X = \{A, B, C, B, D, A, B\}$.
	\\
	\\
	Definição mais formal:\\
	\\
	\textit{Longest Common Subsequence (LCS)}: Datas as sequências $X [1..m]$ e $Y [1..n]$ encontrar uma sequência $Z$ que seja subsequência de $X$ e $Y$ e tem comprimento (número de símbolos) máximo.\\
	\\
	$X = \{A, B, C, B, D, A, B\}$\\
	$Y = \{B, D, C, A, B, A\}$\\
	$LCS (X,Y) = \{B, C, B, A\}$\\

	Como o problema específico neste trabalho é encontrar palíndromos, então nossa sequência $Y$ é o inverso da nossa sequência $X$.

\section{SOLUÇÃO PROPOSTA}
\label{solucao_proposta}

\subsection{Abordagem Dinâmica}
\label{dinamica}

    Uma primeira abordagem escolhida para resolver este problema é o de Programação Dinâmica. Programação dinâmica é um método para a construção de algoritmos para a resolução de problemas computacionais, em especial os de otimização combinatória. Ela é aplicável a problemas nos quais a solução ótima pode ser computada a partir da solução ótima previamente calculada e memorizada - de forma a evitar recálculo - de outros subproblemas que, sobrepostos, compõem o problema original \cite{szwarccfiter}.

    O que um problema de otimização deve ter para que a programação dinâmica seja aplicável são duas principais características: subestrutura ótima e superposição de subproblemas. Um problema apresenta uma subestrutura ótima quando uma solução ótima para o problema contém em seu interior soluções ótimas para subproblemas. A superposição de subproblemas acontece quando um algoritmo recursivo reexamina o mesmo problema muitas vezes \cite{szwarccfiter}.

    O \textit{LCS} tem \textbf{subestrutura ótima}, pois pode ser quebrado em problemas menores, mais simples, os quais também podem ser divididos em subproblemas menores ainda, e assim por diante, até, finalmente, a solução for trivial. O \textit{LCS} também tem \textbf{sobreposição de problemas}: a solução de um subproblema maior depende da solução de vários subproblemas menores \cite{wikilcs}.

    Para calcular o \textit{LCS} utilizamos uma tabela para armazenar os valores calculados a cada passo da execução. As implementações mais encontradas na internet utilizavam uma tabela $(n+1, m+1)$, sendo $n$ e $m$ o tamanho de cada uma das strings. A abordagem dinâmica escolhida aqui foi utilizar uma tabela com apenas $2$ linhas e $n$ colunas, sendo $n$ o tamanho da \textit{string} que encontraremos o menor palíndromo, pois como o trabalho pede apenas o tamanho do \textit{LCS}, então só precisamos do estado atual e anterior de cada coluna da matriz, por isso utilizamos apenas duas linhas para realizar os cálculos \cite{hirsch}.

    A seguir um pseudocódigo do algoritmo implementado:

    \newpage

    \begin{lstlisting}[language=c]
    int LCS(X)
        M = array(n, 2) // Matriz de estados do LCS

        for i = n ate 0
            charRev = X[i] // Le palavra reversamente

            for j = 0 ate n
                charCon = X[j] // Le palavra continuamente

                    if charCon = charRev
                        M[i,j+1] = M[0,j] + 1
                    else
                        M[i,j+1] = max(M[1,j], M[0,j+1])

            for k = 0 ate n
                M[0][k] = M[1][k] // Atualiza estado atual

    return n - M[1,n]
    \end{lstlisting}


\subsubsection{Algoritmo implementado}

\begin{itemize}
 \item \begin{large}\textit{int lcs(char * palavra, int len, Matriz * estados)}\end{large}\\
 \subitem \textbf{Descrição:} Calcula o \textit{LCS} da palavra e retorna seu valor.
 \subitem \textbf{Parâmetros:} Vetor da palavra, inteiro contendo o tamanho da palavra, e uma matriz de estados.
 \subitem \textbf{Complexidade:} $O(n^2)$, onde $n$ é o tamanho da palavra.
\end{itemize}

\subsection{Abordagem Gulosa}
\label{gulosa}

A abordagem gulosa escolhida foi a seguinte: o algoritmo percorre a palavra letra por letra e a cada momento verifica se essa palavra é um palíndromo ou não. Ele percorre a palavra inteira para encontrar o maior palíndromo contido nela, então de acordo com seu tamanho e o tamanho do maior palíndromo, podemos deduzir que a quantidade de letras a serem inseridas para transformar a palavra inicial em um palíndromo.

A cada iteração ele verifica se a atual substring é um palíndromo ou não, caso seja, ela verifica se ela é maior que o palíndromo encontrado anteriormente, se for então é armazenado a quantidade de caracteres desse novo palíndromo encontrado.

O algoritmo não pode ser considerado ótimo, pois ele supõe que o maior palíndromo estará no início ou no final da palavra, em outras palavras, o algoritmo apenas encontra palíndromos que são sufixos ou prefixos da palavra. Então para casos em que o maior palíndromo esteja no meio da palavra o algoritmo guloso não será nem um pouco ótimo.

Logo abaixo segue um exemplo para o algoritmo ficar mais claro:

    \begin{lstlisting}[language=c]
        Palavra: ovobanana
        a 1 // Marca como sendo o maior palindromo
        o 1
        na 0
        vo 0
        ana 1 // Palindromo maior do que o encontrado anteriormente
        ovo 1
        nana 0
        bovo 0
        anana 1 // Palindromo maior do que o encontrado anteriormente
        abovo 0
        banana 0
        nabovo 0
        obanana 0
        anabovo 0
        vobanana 0
        nanabovo 0
        ovobanana 0
        ananabovo 0
    \end{lstlisting}

    Então o algoritmo encontrou o maior palíndromo (anana), que contém 5 caracteres. A resposta que ele retorna é a quantidade de letras da palavra original, menos a quantidade de letras do palíndromo. Nesse caso, o algoritmo irá precisar inserir quatro caracteres para transformar a palavra toda em um palíndromo.

\subsubsection{Algoritmos implementados}

\begin{itemize}
 \item \begin{large}\textit{int palindromo(char *palavra, int len)}\end{large}\\
 \subitem \textbf{Descrição:} Verifica caso a palavra em questão é um palíndromo ou não
 \subitem \textbf{Parâmetros:} Vetor da palavra e um inteiro contendo o tamanho da palavra.
 \subitem \textbf{Complexidade:} $O(n)$, onde $n$ é o tamanho da palavra.
\end{itemize}

\vspace{0.2 true cm}

\begin{itemize}
 \item \begin{large}\textit{int guloso(char * palavra, int len)}\end{large}\\
 \subitem \textbf{Descrição:} Algoritmo que vasculha a palavra para encontrar o maior palíndromo que seja sufixo ou prefixo da palavra.
 \subitem \textbf{Parâmetros:} Vetor da palavra e um inteiro contendo o tamanho da palavra.
 \subitem \textbf{Complexidade:} $O(n^2)$, onde $n$ é o tamanho da palavra.
\end{itemize}

\vspace{0.2 true cm}


\section{IMPLEMENTAÇÃO}
\label{implementacao}

\subsection{Código}

\subsubsection{Arquivos .c}

\begin{itemize}
\item \textbf{tp2pd.c} Arquivo principal do programa com a abordagem de programação dinâmica, chama a função LCS e insere cada resultado no arquivo de saída.
\item \textbf{tp2g.c} Arquivo principal do programa com a abordagem gulosa, chama a função "guloso" e insere cada resultado no arquivo de saída.
\item \textbf{lcs.c} Contém a implementação da função LCS, utilizada na abordagem de programação dinâmica.
\item \textbf{guloso.c} Contém a definição da função guloso, utilizada na abordagem de paradigma guloso, e da função palindromo.
\item \textbf{matriz.c} Contém todas as funções de manipulação, leitura e escrita de matrizes.
\item \textbf{arquivos.c} Um tipo abstrato de dados de manipulação de arquivos, contendo funções de abertura, leitura, escrita e fechamento.
\end{itemize}

\subsubsection{Arquivos .h}

\begin{itemize}
\item \textbf{lcs.h} Contém a definição da função LCS, utilizada na abordagem de programação dinâmica.
\item \textbf{guloso.h} Contém a definição da função guloso, utilizada na abordagem de paradigma guloso, e da função palíndromo.
\item \textbf{matriz.h} Além de definir a estrutura de matriz, contém o cabeçalho todas as funções de manipulação, leitura e escrita de matrizes.
\item \textbf{arquivos.h} Definição da das funções utilizadas para ler, escrever e fechar corretamente um arquivo.
\end{itemize}

\subsection{Compilação}

O programa deve ser compilado através do compilador GCC através dos seguintes comandos

Para programação dinâmica:
\begin{footnotesize}
\begin{verbatim}
gcc -Wall -Lsrc src/tp2pd.c src/arquivos.c src/matriz.c src/lcs.c -o tp2pd \end{verbatim}
\end{footnotesize}

E para o algoritmo guloso:
\begin{footnotesize}
\begin{verbatim}
gcc -Wall -Lsrc src/tp2g.c src/arquivos.c src/matriz.c src/guloso.c -o tp2g \end{verbatim}
\end{footnotesize}

Ou através do comando $make$.

\subsection{Execução}

A execução do programa tem como parâmetros:
\begin{itemize}
\item Um arquivo de entrada contendo várias instâncias de palavras.
\item Um arquivo de saída que irá receber o resultado dos cálculos do menor palíndromo possível a ser gerado.
\end{itemize}

O comando para a execução do programa é da forma:

Programação Dinâmica:
\begin{footnotesize}
\begin{verbatim} ./tp2pd <arquivo_de_entrada> <arquivo_de_saída>\end{verbatim}
\end{footnotesize}

Paradigma Guloso:
\begin{footnotesize}
\begin{verbatim} ./tp2g <arquivo_de_entrada> <arquivo_de_saída>\end{verbatim}
\end{footnotesize}

\subsubsection{Formato da entrada}

A primeira linha do arquivo de entrada contém o valor \textit{k} de instâncias que o arquivo contém. As próximas $k$ linhas contém as palavras.

A seguir um arquivo de entrada de exemplo:

\begin{verbatim}
5
BOLO
XBOX
BANANA
SUCESSO
WOW
\end{verbatim}

\subsubsection{Formato da saída}

O arquivo de saída consiste em k linhas com o número mínimo de caracteres necessários para gerar um palíndromo com a palavra dada de entrada na linha correspondente. Um exemplo é mostrado abaixo:

\begin{verbatim}
1
1
1
4
0
\end{verbatim}


\section{AVALIAÇÃO EXPERIMENTAL}
\label{avaliacao_experimental}

Para testar os algoritmos foi feito o seguinte: um script foi criado que, utilizando um dicionário de aproximadamente 2000 palavras, geraria uma sequência de palavras, concatenadas ou não, para testar cada tipo de cenário.

Foi realizado uma série de testes, com entradas crescentes, da maneira que é explicado nas próximas seções.

\subsection{Máquina utilizada}
\label{maquina}

Segue especificação da máquina utilizada para os testes:
\begin{verbatim}
model name:     Intel(R) Core(TM) i3 CPU       M 330  @ 2.13GHz
cpu MHz:        933.000
cache size:     3072 KB
MemTotal:       3980124 kB
\end{verbatim}


\subsection{Testes}
\label{testes}

Então foi pensado em cinco testes a serem executados, um com apenas uma palavra pra ser testada, outro com 5 palavras concatenadas, um com 10 palavras concatenadas, um com 50 palavras concatenadas e, por último, um teste com 100 palavras concatenadas. Assim teríamos testes com palavras relativamente pequenas e com uma série de caracteres relativamente grandes. Assim poderíamos ver o comportamento de cada paradigma para tanto entradas pequenas quando grandes.

O gráfico que será mostrado em cada teste mostra especificamente a quantidade de caracteres (em porcentagem) a mais que o algoritmo guloso teve de inserir que o algoritmo de programação dinâmica. Assim podemos fazer uma análise entre a solução não-ótima e a solução ótima.

Dentro de cada subseção será dado um exemplo de entrada para cada teste e será detalhado melhor as características de cada um.

\subsubsection{Testes pequenos - Uma palavra}
\label{pequeno}

A quantidade média de caracteres neste teste ficou entre aproximadamente $10$. A seguir um exemplo das palavras utilizadas no teste:

\begin{verbatim}
abbreviations
abandoned
youthfulness
...
\end{verbatim}


    \begin{figure}[h!]
        \centering
        \includegraphics[width=0.8\textwidth]{umapalavra.png}
        \caption{Quantidade de caracteres a mais (em \%)}
        \label{umapalavra}
    \end{figure}

    Na Figura \ref{umapalavra} é perceptível que a grande maioria dos testes não houve diferença entre a solução gulosa e a solução ótima. Porém, quando houve, a diferença de caracteres cresceu bastante, chegando até a $200\%$.

    Para não alterar demais a visualização do gráfico, um resultado foi removido pois a diferença de caracteres era de $800\%$; mas, de uma série de 200 palavras testadas, apenas uma que teve esse comportamento. Isso ocorreu quando o algoritmo de programação dinâmica teve de inserir apenas uma letra e o guloso inseriu 9.


\subsubsection{Testes médios - Cinco palavras concatenadas}
\label{medio}

Na Figura \ref{cincopalavras} pode-se ver o resultado do teste que utilizou uma série de palavras concatenadas, neste caso, cinco palavras. Cada instância ficou com aproximadamente $30$ até $40$ caracteres. A seguir um exemplo das palavras usadas no teste:

\begin{verbatim}
abatedabnormallywondrousxeroxingzero
deifiedwonabolishmentsabbottzenith
yankeewizardwoefulabaseswrangle
...
\end{verbatim}


    \begin{figure}[h!]
        \centering
        \includegraphics[width=0.8\textwidth]{cincopalavras2.png}
        \caption{Quantidade de caracteres a mais (em \%)}
        \label{cincopalavras}
    \end{figure}

\subsubsection{Testes médios - Dez palavras concatenadas}
\label{medio10}

Na Figura \ref{dezpalavras} consta o resultado do teste que utilizou uma série de caracteres que eram formados por palavras concatenadas umas nas outras, neste caso, 10 palavras no total. Cada instância ficou com aproximadamente $60$ até $80$ caracteres. A seguir um exemplo das palavras usadas no teste:

\begin{verbatim}
woveyugoslavianyoungsterreferyuridevovedwonderabashingaboveyellowed
wontwritesabaseszuluwrestyeayesterdayabbreviatingzoologicallyzone
yorktownwrittenaberrationsaibohphobiaababawrappedwolffaberrantablewrote
...
\end{verbatim}


    \begin{figure}[h!]
        \centering
        \includegraphics[width=0.8\textwidth]{dezpalavras3.png}
        \caption{Quantidade de caracteres a mais (em \%)}
        \label{dezpalavras}
    \end{figure}


\subsubsection{Testes grandes - Cinquenta palavras concatenadas}
\label{grande50}

Na Figura \ref{cinquentapalavras} aumentamos a quantidade de palavras concatenadas para cinquenta. Neste teste não iremos mostrar as palavras concatenadas, mas cada uma ficou com $350$ até $450$ caracteres.

    \begin{figure}[h!]
        \centering
        \includegraphics[width=0.8\textwidth]{cinquentapalavras.png}
        \caption{Quantidade de caracteres a mais (em \%)}
        \label{cinquentapalavras}
    \end{figure}

\subsubsection{Testes grandes - Cem palavras concatenadas}
\label{grande100}

Na Figura \ref{cempalavras}, no último teste, concatenamos cem palavras. Não iremos mostrar as palavras concatenadas, mas cada uma ficou com uma média de $900$ caracteres.

    \begin{figure}[h!]
        \centering
        \includegraphics[width=0.8\textwidth]{cempalavras.png}
        \caption{Quantidade de caracteres a mais (em \%)}
        \label{cempalavras}
    \end{figure}

\subsection{Resultado}

	Como é possível perceber pela série de testes na última seção, quanto maior a cadeia de caracteres a ser computada pelos algoritmos, mais homogênea e regular a diferença entre cada solução fica. Nos primeiros teste, de apenas uma palavra \ref{pequeno}, e de 5 palavras \ref{medio}, a variação é gritante. No primeiro varia de $0\%$ até $800\%$, no segundo a variação diminui mas ainda é relativamente grande. Podemos ver na Figura \ref{cincopalavras}, que a quantidade de caracteres vai de aproximadamente $15\%$ para $100\%$.

	Nos últimos testes a curva de variação de caracteres vai se tornando cada vez mais estável. No teste com dez palavras concatenadas, na Figura \ref{dezpalavras}, a variação vai de aproximadamente $25\%$ para $75\%$. Mas quando os caracteres vão para a casa das centenas e milhares, essa variação se torna cada vez menor, se localizando em torno dos $60\%$, $70\%$.


\section{CONCLUSÃO}
\label{conclusao}

    Os dois algoritmos implementados solucionam o problema do palíndromo apresentado. O primeiro algoritmo, utilizando o paradigma de programação dinâmica, possui solução ótima, ou seja, sempre encontrará o melhor resultado. Já o algoritmo guloso, possui solução não-ótima, pois toma as decisões que julga ser melhor naquele momento específico, porém sem considerar vários outros aspectos do problema.

    Foi visto na Avaliação Experimental, que quanto maior a cadeia de caracteres, mais constante ficava a diferença entre as soluções ótima e não-ótima. Claro que a solução ótima ainda conseguia resolver o problema com uma média de $60\%$ a menos de caracteres inseridos, mas ainda assim era um resultado que não era esperado. Acreditava que a solução não-ótima do algoritmo guloso teria um rendimento muito pior.

    Caso fosse acatar o resultado do teste \ref{pequeno}, poderia dizer que o teto observável para a quantidade de caracteres a ser inseridos seria $800\%$, porém esse foi um único caso dentre centenas de outros testados. Observando os resultados dos testes maiores, poderia dizer que o teto observável ficaria em torno de $75\%$.

\bibliographystyle{sbc}
\bibliography{tp2}

\end{document}
